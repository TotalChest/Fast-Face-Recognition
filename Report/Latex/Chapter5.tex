\section*{Заключение}
\addcontentsline{toc}{section}{Заключение}
\label{sec:Chapter5} \index{Chapter5}
В рамках данной работы были исследованы современные индексные структуры быстрого поиска информации в больших объемах данных, а также проверена эффективность этих структур в задаче распознавания лиц. В результате экспериментов было выяснено, что при использовании любой из индексных структур скорость поиска возрастает в сотни раз по сравнению с исчерпывающим поиском. Временные затраты поиска в иерархической и мульти-индексной структурах находятся в окрестности нескольких миллисекунд, что согласуется с постановкой. Как и ожидалось, в ANN алгоритмах точность поиска начинает падать. При наивном использовании мульти-индексной структуры, точность поиска существенно снижается, что делает невозможным применение данного алгоритма на практике. Однако некоторые виды предобработки данных способны решить эту проблему. По точности и времени распознавания иерархическая структура оказалась наиболее универсальной.

Подводя итоги, приходим к выводу: в больших коллекция изображений лиц индексные структуры могут показать приемлемую скорость поиска с хорошим показателем точности, что дает им право использоваться в реальных задачах распознавания.