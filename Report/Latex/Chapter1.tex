\section{Постановка задачи}
\label{sec:Chapter1} \index{Chapter1}

Главной задачей данной работы является исследование современных индексных структур быстрого поиска и поверка их эффективности в задаче распознавания лиц. Считается, что выделением лица на изображении и построением его признаков занимаются алгоритмы общедоступных сверточных нейронных сетей. Работа производится в рамках построенных 128-мерных векторов, по одному для каждого изображения. Время построения каждого вектора учитывать не будем, так как во всех алгоритмах будет использоваться одна и та же нейронная сеть. Также не будем учитывать, но обратим внимание на время обучения индексных структур, так как для разных алгоритмов оно может отличаться на порядки.

Проверка заключается в измерении среднего времени поиска похожих лиц в большой коллекции изображений. Время, затраченное на поиск $K$ ($K = 1, 5, 10, 30, 50, 100$) ближайших соседей является основным критерием скорости алгоритма. Помимо скорости важно учитывать точность поиска. Во всех сопутствующих работах \cite{1,3,5,6,7} точность измеряется как процент истинных ближайших соседей среди $K$ найденных, где истинные ближайшие соседи определяются точным евклидовым расстоянием. Использовать данную метрику в этой работе следует аккуратно, так как она не отражает природы исследуемой области, а указывает только лишь на качество используемого алгоритма, который и без того много лет тестируется и улучшается. В связи со спецификой нашей задачи, наиболее правильным вариантом измерения точности будет получение процента лиц, отмеченных так же, как и запрос, среди $K$ найденных ближайших соседей. Также в качестве альтернативного показателя качества будем измерять частоту вхождения каждого лица в $K$ ближайших соседей. В случае совпадения лица запроса и лица с наибольшей частотой будем говорить об успешном распознавании. Во многих задачах распознавания образов используется именно эта метрика.

Для применимости данного подхода к реальным задачам алгоритм должен удовлетворять нескольким требованиям:
\begin{enumerate} 
\item Скорость поиска в большом объеме изображений не должна превышать нескольких миллисекунд;
\item Точность поиска должна быть в пределе допустимой для выбранного алгоритма, то есть не сильно отличаться от приводимой в статьях.
\end{enumerate} 

В качестве вывода следует оценить показатели точности и скорости распознавания лиц на основе результатов экспериментов, а также сформулировать возможные пути улучшения.
