\section*{Введение}
\addcontentsline{toc}{section}{Введение}
\label{sec:Chapter0} \index{Chapter0}


В последнее десятилетие визуальный поиск стал широко распространенной функцией многих поисковых систем. Большой объем сохраненных изображений в сети измеряется петабайтами, число которых с каждым днем только увеличивается. В связи с этим эффективный поиск ближайших соседей является серьезной исследовательской проблемой~\cite{1,2,3,4,5,6}. Потребность быстрого поиска похожих изображений занимает большую нишу в современных приложениях компьютерного зрения, в том числе в задаче распознавания лиц~\cite{8}. Социальные сети, правоохранительные органы имеют огромные коллекции изображений лиц, среди которых надо уметь быстро извлекать нужную информацию. Для решения данной проблемы требуются эффективные и масштабируемые алгоритмы поиска с низкими временными затратами. Ожидается, что ответ на запросы к базам данных из миллиардов элементов будет занимать несколько миллисекунд.

Решение данной задачи можно рассматривать с двух сторон. Во-первых, даже самые современные алгоритмы поиска и обработки лиц на изображении не идеальны. Это открывает просторы для исследований. Во-вторых, подходящая структура данных для поиска может давать многократное увеличение скорости. Среди всех алгоритмов распознавания сверточные нейронные сети (CNN) показывают лучшие результаты поиска лиц на изображении и используются в большинстве исследованиях этой области \cite{8,10}. В связи с этим основной задачей исследования будем считать проблему выбора поисковой структуры данных, а дескрипторы лиц для экспериментов будем строить по одной из общедоступных CNN. 

Все существующие крупномасштабные поисковые системы избегают исчерпывающего поиска путем ограничения конечного набора кандидатов, который рассматривается для запроса. Данный подход называют приближенным поиском ближайших соседей (ANN). Современные алгоритмы ANN имеют три основных реализации: инвертированная индексация~\cite{1,2,3,4,5,6}, хеширование~\cite{9}, многомерная инвертированная индексация, основанная на квантовании произведения (PQ)~\cite{4,5}. В этой работе основное внимание уделено инвертированному индексу и его оптимизации с помощью PQ.

Структуры индексации разбивают пространство поиска на большое количество непересекающихся областей, и в процессе поиска используется только малая часть коллекции, наиболее близкая к конкретному запросу. Отобранная часть данных образует короткий список кандидатов, и поисковая система рассчитывает расстояния между запросом и всеми кандидатами. На этом этапе важно, чтобы список кандидатов был коротким, так как вычисление расстояния имеет линейную сложность по данной длине. Метод PQ для ANN используется в двух видах: для построения многомерного инвертированного индекса для приближенного поиска или для кодирования векторов в компактные коды для точного поиска. Идея этих подходов состоит в том, чтобы разложить пространство векторов на большое количество непересекающихся множеств и обучить запросы получать доступ к ближайшим из них.

Первая структура индексации, способная работать с миллиардным набором данных, представлена в~\cite{1}. Она основана на структуре инвертированного индекса, которая разбивает пространство признаков на диаграмму Вороного. Каждая область задается своим центроидом, который предварительно обучили алгоритмом $K$-средних. Показано, что эта система достигает разумных скоростей поиска, порядка нескольких десятков миллисекунд. Позже обобщение структуры инвертированного индекса предложили в~\cite{3}. В этой работе представлен инвертированный многомерный индекс или мульти-индекс (IMI), который разбивает пространство признаков на несколько ортогональных подпространств, и каждое подпространство обучается независимо друг от друга. Декартово произведение такого разбиения образует неявное разбиение всего пространства поиска. Обе эти структуры обладают своими недостатками, которые можно устранить с помощью различных оптимизаций PQ~\cite{6,7}.

В данной работе описано несколько современных архитектур индексирования, и путем экспериментов исследована их применимость к задаче распознавания лиц.